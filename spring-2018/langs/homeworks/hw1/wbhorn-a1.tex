%!TEX TS-program = xelatex
%!TEX encoding = UTF-8 Unicode
%!TEX spellcheck
%!TEX options = -shell-escape
% Change the following line to whatever your file is
%!TEX root = ./project0.tex

\documentclass{article}
\author{William Horn}
\title{CS 331 Programming Languages - Assignment 1}
\pagestyle{headings}
\usepackage{amsmath}

\begin{document}
\maketitle

\section*{Exercise A}
\textbf{Secret message} \\

BE SURE TO DRINK YOUR OVALTINE

\section*{Exercise B}
\flushleft
\begin{enumerate}
    \item Is the type checking in C++ primarily static or dynamic?\\
    \vspace{1mm}
    \textbf{Static}

    \item What does this mean?\\
    \vspace{1mm}
    \textbf{Types are checked before the program is run. For C++ this is at compile time.}
\end{enumerate}

\section*{Exercise C}
Consider the following grammar, whose start symbol is \emph{S}.
\begin{align}
S \rightarrow fAB \\
A \rightarrow x | xA \\
B \rightarrow Bo | \epsilon
\end{align}

Which of the following strings are in the language generated by this grammar?  (\emph{fxo}, \emph{xo}, \emph{fooo}, \emph{fxxx}, \emph{fooxx}, \emph{fxoooo}, \emph{fxoxo}) \\
\vspace{1mm}
\textbf{1, 3, 6, 7}
\newpage

\section*{Exercise D}
Give an English description of the language generated by the following grammar, whose start symbol is \emph{S}. Your description should be precise enough that, based only on your description, I can determine which strings lie in the language and which do not.

\begin{align}
S \rightarrow xSz | A \\
A \rightarrow Ayy | \epsilon
\end{align}

\textbf{An equal number of 1 or more x's and z's with a positive even number of y's, including zero, in between them.}

\section*{Exercise E}
Consider the following regular expression.
\begin{equation}
(ab | c)*x*
\end{equation}

Which of the following strings are matched by this regular expression? (\emph{aaax, aabbxx, abcxx, abc, abab, baba, cab})\\
\vspace{1mm}
\textbf{3, 4, 5, 7}

\section*{Exercise F}
Write a regular expression that matches words that contain at least one \emph{b}, and no letters other than lower-case \emph{a} and \emph{b}. (Here, a word is any sequence of letters.)

\begin{equation}
\textbf{ [ab]*b[ab]* }
\end{equation}

\section*{Exercise G}
This problem deals with the following grammar, which has start symbol \emph{S}.

\begin{align}
S \rightarrow SS | A \\
A \rightarrow xy
\end{align}

\begin{enumerate}
    \item Give a leftmost derivation for the string \emph{xyxy}. \\
        \vspace{1mm}
         \textbf{S},
         \textbf{SS},
         \textbf{AS},
         \textbf{xyS},
         \textbf{xyA},
         \textbf{xyxy}
    \item Give a rightmost derivation for the string \emph{xyxy}. \\
        \vspace{1mm}
         \textbf{S},
         \textbf{SS},
         \textbf{SA},
         \textbf{Sxy},
         \textbf{Axy},
         \textbf{xyxy}
    \item Verify that the grammar is ambiguous by finding a string with two different parse trees. Your answer should consist of the string and the two parse trees.

\begin{verbatim}
    xyxyxy                                xyxyxy

      S                                     S
    /    \                                /    \
   /      \                              /      \
  S        S                            S        S
  |       /  \                         /  \      |
  |      /    \                       /    \     |
  A     S      S                     S      S    A
 / \    |      |                     |      |   / \
x   y   A      A                     A      A  x   y
       / \    / \                   / \    / \
      x   y  x   y                 x   y  x   y
\end{verbatim}

    \item Write a non-ambiguous grammar that generates the same language as the above grammar.
\begin{align}
S \rightarrow xyS | A \\
A \rightarrow xy | \epsilon
\end{align}
\end{enumerate}

\section*{Exercise H}
\begin{equation}
\{axx, axxx, axxxx, axxxxx, \ldots \}
\end{equation}

\begin{enumerate}
\item Write a regular expression that generates this language.
\begin{equation}
\textbf{axx(x*)}
\end{equation}
\item Draw a diagram of a DFA that recognizes this language.
\begin{verbatim}
                       +----<--+
                       |       |
           /-\  axx   /-\      | x
start --->|   | ---> |   | ->--+
           \-/        \-/


\end{verbatim}
\item Write a BNF grammar that generates this language.
\begin{verbatim}
<grammar> ::= "axx" <more-xs>
<more-xs> ::=  <more-xs> <more-xs> | "x"
\end{verbatim}

\item Is your grammar from the previous step ambiguous? Justify your answer.\\
\vspace{1mm}
\textbf{Either the right or left more-xs can be expanded, this makes the grammar ambiguous.}
\end{enumerate}
\end{document}
