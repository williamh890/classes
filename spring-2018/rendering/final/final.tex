\documentclass{article}

\usepackage[italic]{mathastext}
\usepackage{amsmath}
\usepackage{amssymb}


\begin{document}

\newcommand{\vect}[1]{\mathbf{#1}}

\title{\textbf{Graphical Rendering Final}}

\author{
  William Horn\\
  \texttt{wbhorn@alaska.edu}
}
\date{\today}
\maketitle

\section{Ray Tracing}

\begin{itemize}
    \item[\textbf{(a)}] Whitted's paper solved accurately Simulating refraction, shadows, and reflections
    \item[\textbf{(b)}] With using traditional rendering techniques, these shapes are approximated using
        triangles. Ray marching allows for these shapes to be rendered exactly using there mathematical.
        Also reflections, refractions, and shadows can be naturally rendered. With traditional methods
        these effects have to be added and don't come free with the technique.
    \item[\textbf{(c)}] The ray marching algorithm can speed up ray tracing by marching rays from both
        the viewpoint and the light sources and then connecting the viewpoint rays to the light rays.
\end{itemize}

\section{Research}

\subsection{Whitted}

Allowed for realistic looking scenes to be rendered using ray tracing to improve visual fidelity with
shadows, refractions and reflections. It allowed for light and objects to more realistically interact with
one another within the scene. This is following the rays recursively as they interact throughout the scene,
just like real light rays would.

\subsection{Blinn}

Computationally cheap method for rendering surfaces other than smooth ones. While previous methods just tried
to slap a wrinkly texture onto the surfaces, Blinn's technique was to move the normals themselves to change the
lighting of the surfaces. The values used to perturb the normals were look up from a texture to be computationally
efficient.

\subsection{Hart et al.}

Provided a more computationally efficient way of rendering complex mathematical structures.
Previously only surfaces with Euclidean function could be rendered efficiently, this did not
include fractals.

\subsection{Cook and Torrance}

This paper was to add some softness to ray traced images. Previously shadows, reflections and
edges were all sharp and exact. By distributing rays temporally, these sharp edges are softened.
Effects such as motion blur and depth of field can be implemented using distributed ray tracing.

\subsection{Kajiya}

The rendering equation combined a collection of algorithms that were attempting to simulate
the physical phenomena of light bouncing off of an object. It put into context the little
pieces and parts into something that could be applied more generally to different types of
surfaces.

\subsection{Perlin}

Method for generating smooth random numbers. A giant leap forward in procedural generation
of textures. Great for generating textures like wood and marble. It can even be used for creating
3d objects like clouds/fire or level and terrain generation. Pretty much the swiss army knife
of procedural generation.

\subsection{Veach and Guibas}

Applies different Monte Carlo sampling techniques to produce a lower variance, higher quality
image. With any Monte Carlo integration, the output is going to be non-exact because of the
randomness inherent in the method. This aim is to reduce the variance and improve the overall
quality of the output by weighting the samples from multiple techniques.

\subsection{Gooch et al.}

Uses non-photo realistic rendering to accentuate geometric information using color and lighting.
It is an attempt to mimic professionally drawn technical drawings and is a substitute for
the Blinn Phong Lighting model.

\section{Path Tracing}

\begin{itemize}
    \item[\textbf{(a)}] approximated 1.32
    \item[\textbf{(b)}] Yes, my guess in the difference is that because this is an exponential function
        the higher values have a greater output on the approximation then the lower ones. The samples
        given have $\mu = 1.037$ and $\sigma = 0.524$, meaning that the both small and large values are
        equally represented. This seems like a logical way to sample values, but because larger values have a
        greater weight on the final answer, these samples lead to an incorrect answer using Monte
        Carlo method.
    \item[\textbf{(c)}] Instead of sampling multiple times at the same moment in time, the rays through
        a range of time are combined. Depth of field, motion blur effects can be implicated using distributed
        ray tracing.
\end{itemize}

\section{Physically Based Rendering}

\begin{itemize}
    \item[\textbf{(a)}] An equation for determining the color of a surface.
$$
    \mathbf{L}_o(\mathbf{x} \rightarrow{} \omega{}_o) =
    \mathbf{L}_e(\omega{}_o \rightarrow{} \mathbf{x}) +
    \int_{\Omega{}}^{} fr (\omega{}_i, \omega{}_o)
    \mathbf{L}_i (\omega{}_i \rightarrow{} x)
    \langle{}\omega{}_i, \omega{}_g\rangle{}
    d_\omega{}
$$

\begin{itemize}
    \item[$L_o$] Light out at an angle
    \item[$L_e$] Light emitted from the point
    \item[$\int_{\Omega{}}^{}$] Take into account all angles
    \item[$L_i$] Light coming in from
    \item[$fr$] bidirectional reflectance distribution function
    \item[$ \langle{}\omega{}_i, \omega{}_g\rangle{} $] positive cosine for Glancing angles
    \item[$ d_\omega{} $] Take all angles of incoming light into account
\end{itemize}

\item[\textbf{(b)}] The color at incident angles is what the fresnel reflectance controls. If the
    color on the display is not calibrated correctly the picture could look wrong. Absorptive materials
    don't reflect light uniformly so there is no specular. Dielectrics have both reflections and
    refractions with incoming light.

\end{itemize}

\section{Ray Marching}

\begin{itemize}
    \item[\textbf{(a)}]
        \begin{itemize}
            \item[\textbf{a.}] UNION $\rightarrow{} min(d_1, d_2)$
            \item[\textbf{b.}] INTERSECTION $\rightarrow{} max(d_1, d_2) $
            \item[\textbf{c.}] SUBTRACTION $\rightarrow{} max(d_1, -d_2) $
        \end{itemize}

    \item[\textbf{(b)}]
        \begin{itemize}
            \item[\textbf{a.}] SPHERE $\rightarrow{} f(\vect{p}) = ||\vect{p}|| - 1$
            \item[\textbf{b.}] BOX $\rightarrow{} f(\vect{p}, \vect{b}) = ||\max{(0, ||\vect{p}||-\vect{b})}||$

            \item[\textbf{c.}] TORUS $\rightarrow{} f(\vect{p}, r_x, r_y) = ||(||p_{xz}|| - r_x, p_y)|| - r_y$

            \item[\textbf{d.}] PLANE $\rightarrow{} f(\vect{p}, \vect{n}, d) = \vect{p} \cdot \vect{n} + d$

            \item[\textbf{e.}] CYLINDER $\rightarrow{} f(\vect{p}, \vect{c}) = ||p_{xz} - c_{xy}|| - c_z$

            \item[\textbf{f.}] CONE $\rightarrow{} f(\vect{p}, \vect{c}) = \vect{c} \cdot (||p_{xy}||, p_z)$
        \end{itemize}
    \item[\textbf{(c)}] In the negative sign means that of the 2 signed distances, the one that gets subtracted
        gets taken out of the overall shape.
\end{itemize}

\section{Nonphotorealistic Rendering}

\begin{itemize}
    \item[\textbf{(a)}] Sobel, Prewitt,  Robert's cross are all edge detection methods
    \item[\textbf{(b)}] Toon Shading $\rightarrow$ X-Toon, CAD Technical Illustrations $\rightarrow$ Gooch Shading
    \item[\textbf{(c)}] Use banded shading to approximate painterly
\end{itemize}

\section{Procedural Generation}

\begin{itemize}
    \item[\textbf{(a)}] Perline Noise
    \item[\textbf{(b)}] Procedural generation is a method of creating data algorithmically as opposed to manually.
    \item[\textbf{(c)}] User Perlin noise to generate positions for objects in a procedurally generated world. Objects
        would be things like foliage, trees, rocks and other terrain. Also noise could be used to create a height map to give the
        world dimension.
\end{itemize}

\end{document}
