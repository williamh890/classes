\documentclass{IEEEtran}

\usepackage{graphicx}
\usepackage{amsmath}
\usepackage{refstyle}
\def\RSfigtxt{Fig.\,}%
\def\RSfigstxt{Figs~}%
\def\RSFigtxt{Fig.\,}%
\def\RSFigstxt{Figs~}%

\begin{document}

\title{Improvements on Realistic Snow Shader}
\author{William Horn\\University of Alaska Fairbanks\\wbhorn@alaska.edu}
\date{\today}
\maketitle

\section{Introduction}

Snow is a challenging substance to simulate and render because of its highly
dynamic nature. It is hard enough to render a totally static snowy scene, but
adding onto of that the way wind interacts to create complex macro features. Add
to this the handling of the state transfers (snow to ice, snow to water, water
to ice) that can occur and change the physical and visual properties and
simulating this in real time with today's technology is not feasible. In this
paper only a subset of this problem will be tackled; improving on a general use snow
shader \figref{original}.

\begin{figure}
    \includegraphics[width=\linewidth]{images/original.jpg}
    \caption{Original dynamic snow texture}
    \label{fig:original}
\end{figure}

\section{Previous Research}

To generate a realistic snow texture, there are many components that need to be
considered. The previous technique took three into account:
\begin{itemize}
    \item Glint i.e. Snowflake level interaction.
    \item Larger features like snow drifts
    \item The reflection of surrounding light.
\end{itemize}

\subsection{Snowflake}
The first component is the interaction of light with each snowflake at the micro
scale called glint. This is the sparkle of the snow. Whenever the view of the
viewer changes, the snow glints creating a sea of sparkles at varying degrees of
brightness and densities.  There were two main challenge when trying to
implement the glint effect, granularity and changing the brightness of the glint
with movement.

These effects are from the normals of each individual snowflake being different.
To simulate these effects semi random normals generated with a GLSL
implementation \cite{noiseglsl} of Perlin noise
\cite{wiki:perlin} were used in the specular component of the Blinn Phong model
\cite{blinn1977models}. The noise was used to generate a smooth vector field of
random normals. Each normal took 2 Perlin random numbers, one for
the vertical direction and another for the vertical origination.

In retrospect using Perlin noise to generate the normals was a mistake. The
reasoning was that Perlin noise is a method for getting a deterministic random
pseudo random number inside the shader as an attempt to get the smooth phasing
in and out with movement. This however would have been taken care of by the
Blinn Phong model \cite{blinn1977models}. The Perlin technique can accomplish
this effect but it doesn't allow for fine grained control over the size of the
sparkle and the speed of the phasing effect.

\subsection{Macro Features}

The next component is getting larger variations in the surface of the snow.
This can either be caused by wind or the sun melting the snow unevenly to create
a rough uneven texture. The previous method simulated the snow melting unevenly and
does not take into account drifting.

To get around the fact that the diffuse component was being ignored a texture
was used in its place. Perlin noise was again used to generate a \(\frac{1}{f}\) noise
pattern to simulate the roughness in the snow \figref{fnoise}.  This created a
nice rough texture that was used to simulate both the large, and the small
deformities in the snow.

\begin{figure}
    \includegraphics[width=\linewidth]{images/fnoise.jpg}
    \caption{Plain \(\tfrac{1}{f}\) noise pattern from original implementation.}
    \label{fig:fnoise}
\end{figure}

\section{Background Problem}

Snow is part of a category of materials called multiscale materials
\cite{multiscale}. These are materials that have micro level details that
effect look of a material. These types of materials are challenging because
accurately rendering the effects of such minute detail often leads to techniques
that are not feasible for real time rendering. It is also a challenge to have a
technique which is flexible enough to be used in a production environment where
the aesthetic outcome of the method will precede the physical accuracy of it
\cite{sparkle}. All these factors must be taken into account.

In practice a large number of techniques end up being used to get the desired
effect. The sand rendering techniques in the video game Journey \cite{journey} is a perfect
example of this. To get the desired outcome for the rendering
of sand took many layers of different techniques and methods. All the while
performance was keep in mind to allow for real time rendering.

A common technique for improving performance is to pre-computed textures instead of
doing calculations at runtime \cite{journey}. Instead of having to do the same calculation
every frame, it can be done once, and referenced throughout the programs
runtime. With normals being calculated using perlin noise at runtime, a large
portion of the calculation for each frame is repeated work.

\section{Proposed Solution}

While there are many improvements that are possible there are three that seem
the most feasible that will, in the end, provide the most value

\subsection{Improved Glint Algorithm}

The original use of using normals generated using perlin noise lead to a visual
outcome that was not artistically flexible or efficient enough to fit the needs of the
project. A better algorithm \cite{sparkle} will be used to
create the micro glints caused by variations in the orientations of the flake
across the surface of the snow. This algorithm intersects the surface that the
glints will appear on with a field of glints. These glints are randomly offset
to give a nonuniform distribution of glints. Measures are also taken to make
sure that the glints have a non-distorted shape and are sparkly even on distant
objects.

\subsection{Textures}

Instead of calculating random numbers every frame, and even every pixel, a noise
texture can be generated prior to runtime and used to lookup the random numbers
that will be needed for the new sparkle algorithm. To make the \(\frac{1}{f}\) noise
pattern which requires multiple levels of detail, multiple noise textures can be
used to get the layered, detailed rough look that is required. Also a texture
will be used to get better fidelity on larger features in the snow like drifts that
are not accurately simulated by the \(\frac{1}{f}\) noise pattern.

\section{Conclusion}

\subsection{Texture}

\subsubsection{Implementation}

May textures were tried to get the look of the snow looking more realistic. A problem
that was encountered was that the WebGL code being used to upload textures only took
png's with square dimensions that were powers of 2. The code for loading/displaying the
textures before was badly tangled into the previous code, so a big portion of the
implementation time was spent cleaning up this code and figuring out how to cleanly
implement this.

Also a lot of time was spent trying different textures that were into powers of two.
Using the algorithm on the Mozilla docs \cite{webgl-tex} only square images with powers
of two dimensions can be used. Images with in png format are used to store the textures.
Prior to the project the code for creating the texture was duplicated in the objects
responsible for rendering the moon and the snowflakes, but now that code is factored out
into a separate function. There is still a small amount of duplication in uploading the
texture to the shader. In the future this could be solved using polymorphism but
that seems to be a little out of the scope of this project.

\subsubsection{Results}

The snow texture does look far better then the  \(\frac{1}{f}\) noise even though it is being tiled,
its detailed enough so that no patterns are easily visible. The texture is 512x512 and
get repeated in the X and Y directions to tile the texture across the entire material.
Applying the snow shader to a object with some dimesion would be an interesting
improvement to see who the shader holds up in different scenarios.

Another improvement the texture gave was performance. The \(\frac{1}{f}\) noise that was being generated
every frame in the shader was not very efficient as it required multiple perlin noise numbers
to be calculated for every fragment.  Using a snow texture for this saves clock cycles and makes
the overall snow shader more efficient. A downside to this is that it is less dynamic because
the texture is tiled across the entire material, in this case, a flat square representing the ground.

\subsection{Glint Algorithm}

\subsubsection{Implementation}

To try and get a more flexible shader a different glint algorithm was implemented. Originally
the glint algorithm from \cite{sparkle} was attempted, but the paper did not go into
enough technical details to easily implement the algorithm, so instead the algorithm
the paper improved on was used instead \cite{shopf}. This was a good way to incrementally
develop because shopf was much simpler.

The main challenge with the algorithm was the number of parameters and constants
that were not explained in the paper itself \cite{sparkle}. With more time it would
be possible to get it to work but a good intermediate point was the original shopf.

\subsubsection{Results}

The original shopf was a simple combination of the 3D noise and the view vector. There
were 5 constants that were not explained but it was very helpful to hook up sliders to
the parameters to change the values. While the glints are nicely shaped with no noise
they were in a grid. The way the noise was applied to the glints actually changed there
shape. One improvements applied to the algorithm in \cite{sparkle} was to apply the noise
to the center point of the glint instead of the glint as a whole.

\figref{shopf-sparkle} shows the results with the Shopf sparkle and the snow texture.
The color of the fog is also mixed with the sparkle to prevent sparkles from showing up
inside of the fog. I still prefer the original sparkle implementation but it was a step
in the right direction.

One thing that was nice about Shopf was that it was formatted as slides as apposed to a
technical paper. Although the technical details about the method were nice in \cite{sparkle}
they were not helpful in applying there solution. More effort should be made in explaining the
applied solution. This may be in a separate document like a blog post or a presentation,
but the academic paper medium makes it very hard to pull out the relevant details needed
to apply the method.

\begin{figure}
    \includegraphics[width=\linewidth]{images/results.png}
    \caption{Using Shopf sparkle algorithm and snow texture}
    \label{fig:shopf-sparkle}
\end{figure}


\bibliographystyle{alpha}
\bibliography{project0}

\end{document}
