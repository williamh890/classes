%!TEX TS-program = xelatex
%!TEX encoding = UTF-8 Unicode
%!TEX spellcheck
%!TEX options = -shell-escape
% Change the following line to whatever your file is
%!TEX root = ./project0.tex
% -*- program: xelatex -*-
% chktex-file 1
% Comments start with a percent sign

% To use ACM Article Format
% \documentclass{acmart}
% \setcopyright{none}

% To use IEEE Transactions format
% \documentclass{IEEEtran}

% To use default article style
\documentclass{article}

\begin{document}

\title{Generating Realistic Looking Snow Drifts}
\author{William Horn\\University of Alaska Fairbanks\\wbhorn@alaska.edu}
\date{\today}
\maketitle

\section{Introduction}

Last semester in graphics I focused on making a semi realist fragment shader for
generating a dynamic snow texture with perlin noise \cite{wiki:perlin}. This looked good but it
was just being rendered on a completely flat plan. To improve this project I
plan on adding some volume to the snow drifts.

\section{Previous Research}

To initially make realistic snow I used a glsl implementation of perlin \cite{noiseglsl} to generate random
normal vectors. These were used to calculate the specular highlight in the blinn phong illumination
\cite{blinn1977models}. Model to create a sparkle when the user moved around the screen.
To make the snow have some texture I used perlin noise to generate a 1/fnoise pattern to use
as the base texture that the sparkle would go on.

\section{Background Problem}

\section{Proposed Solution}

Talk about what you are going to do/did

\section{Results and Analysis}

Talk about your results here and why they are important.

\section{Conclusion}

Talk about your conclusion

% myresearch is the file myresearch.bib where you copy/paste all your BiBTeX references
\bibliographystyle{alpha}
\bibliography{project0}

\end{document}
