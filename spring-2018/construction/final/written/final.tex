\documentclass{article}

% Python code stuff
\usepackage{color}
\usepackage{listings}
\usepackage{setspace}
\definecolor{Code}{rgb}{0,0,0}
\definecolor{Decorators}{rgb}{0.5,0.5,0.5}
\definecolor{Numbers}{rgb}{0.5,0,0}
\definecolor{MatchingBrackets}{rgb}{0.25,0.5,0.5}
\definecolor{Keywords}{rgb}{0,0,1}
\definecolor{self}{rgb}{0,0,0}
\definecolor{Strings}{rgb}{0,0.63,0}
\definecolor{Comments}{rgb}{0,0.63,1}
\definecolor{Backquotes}{rgb}{0,0,0}
\definecolor{Classname}{rgb}{0,0,0}
\definecolor{FunctionName}{rgb}{0,0,0}
\definecolor{Operators}{rgb}{0,0,0}
\definecolor{Background}{rgb}{0.98,0.98,0.98}
\lstdefinelanguage{Python}{
numbers=left,
numberstyle=\footnotesize,
numbersep=1em,
xleftmargin=1em,
framextopmargin=2em,
framexbottommargin=2em,
showspaces=false,
showtabs=false,
showstringspaces=false,
frame=l,
tabsize=4,
% Basic
basicstyle=\ttfamily\small\setstretch{1},
backgroundcolor=\color{Background},
% Comments
commentstyle=\color{Comments}\slshape,
% Strings
stringstyle=\color{Strings},
morecomment=[s][\color{Strings}]{"""}{"""},
morecomment=[s][\color{Strings}]{'''}{'''},
% keywords
morekeywords={import,from,class,def,for,while,if,is,in,elif,else,not,and,or,print,break,continue,return,True,False,None,access,as,,del,except,exec,finally,global,import,lambda,pass,print,raise,try,assert},
keywordstyle={\color{Keywords}\bfseries},
% additional keywords
morekeywords={[2]@invariant,pylab,numpy,np,scipy},
keywordstyle={[2]\color{Decorators}\slshape},
emph={self},
emphstyle={\color{self}\slshape},
%
}
\linespread{1.3}
% End python code stuff

\usepackage{csquotes}
\usepackage{hyperref}

\begin{document}

\title{Software Construction Final}
\author{William Horn\\University of Alaska Fairbanks\\wbhorn@alaska.edu}
\date{\today}
\maketitle

\section{Proxy Design Pattern Implementation}

the github url \url{}

\section{Software Design Patterns}

\subsection{Interpreter}

Define a representation within a language to another one. One example is the python library sqlalchemy \cite{sqlalchemy},
which is a language to interface/generate sql without having to ever write it directly.

While writing the sql into the python as strings, or loading the statements from a file does work, the library provides
a layer of abstraction that can be useful. For instances, if the syntax of sql changed, only sqlalchemy would have to
change, and the client code could stay exactly the same.

\subsection{Strategy}

This decouples implementation from the outcome. A lot of the time we don't care how something gets done, we only care
about the outcome. Making a strategy with a defined interface makes the implementation decoupled from the outcome.
The implementation details can be hide behind a layer of abstraction, making the client of the strategy more clear
and focused.

\subsection{Composition}

\begin{displayquote}
Composite is a structural design pattern that lets you compose objects into tree structures and allow clients to work with these structures as if they were individual objects. \cite {guru}
\end{displayquote}

A perfect example of this is file/folder paradigm used by all major OS's. The same operations can be performed
on both files and folders. Files are the leaf nodes, and folders are the tree structure that can be \emph{composed}
of either/or files and folders.

\subsection{Visitor}

This allows the same operation to be performed on the elements of an object structure. Extending the file/folder example \emph{open}
is a way of visiting. Opening a file will show its contents while opening a file will show another level down in the object tree
structure. The visitor pattern implements double dispatch -- picking the right thing to do given 2 objects. In c++ this can be
done with polymorphic objects or function overloading based on parameter types.

\subsection{Template}

\begin{displayquote}
Define the skeleton of an algorithm in an operation, deferring some steps to client subclasses. \cite{guru}
\end{displayquote}

This allows for processes that are very similar to share code. One example is a running some processing
with different algorithms on some raw data. Each algorithm needs to download data, prep it for the processing,
run the process, then package up the results. Each step is extremely similar with only annoying differences.
The template design pattern can be used to lay out the structure and allow the small differences to be filled in
on a case by case basis.

\subsection{Observer}

\begin{displayquote}
Define a one-to-many dependency between objects so that when one object changes state, all its dependents are notified and updated automatically. \cite{guru}
\end{displayquote}

A great library for this is RxJS \cite{rxjs}, which thoroughly implements the observer patter. Common
applications of this are reacting to user events, notifying subscribers when an http request comes in
or when the state of an application get updated. Programming reactively is an entirely different paradigm
but makes solving certain problems much simpler and clear.

\subsection{Decorator}

\begin{displayquote}
Decorator is a structural design pattern that lets you attach new behaviors to objects by placing them inside wrapper objects that contain these behaviors. \cite{guru}
\end{displayquote}

Decorators in python allow the functionality of functions/methods to be easily extended without. Examples of decorators are one that
times how long a function takes. This doesn't depend on the behavior of the function itself so the timing can be wrapped around the
function with a decorator. Another use case is only allowing a function to be called if a user is authenticated.

\subsection{Command}

\begin{displayquote}
Encapsulate a request as an object, thereby letting you parametrize clients with different requests, queue or log requests, and support undoable operations. \cite{guru}
\end{displayquote}

Using commands as a way of limiting the ways state changes in an application makes the behavior of a complex system well
defined and easy to predict. Also because a commands are well defined the can be logged, making bugs easy to track down,
because the state of the application could have only changed in the commands implementation code. Redux \cite{redux} is a js
front end design framework/patter for managing state in front end web applications. State is immutable and only commands
(called reducers in redux) can create new states by copying the old state with the changes of the command applied.

\subsection{Null Object}

\begin{displayquote}
The intent of a Null Object is to encapsulate the absence of an object by providing a substitutable alternative that offers suitable default do nothing behavior. In short, a design where ``nothing will come of nothing'' \cite{guru}
\end{displayquote}

Null objects are dangerous because they can lead to really pesky bugs, but they can be used in certain
situations to make clean solutions. Binary Tree implementations benefits from null objects because nodes can either be
nodes containing actual data, or null objects, signaling a leaf node.

\subsection{Adapter}

\begin{displayquote}
Adapter is a structural design pattern that allows objects with incompatible interfaces
to collaborate.  \cite{guru}
\end{displayquote}

A good programming patter is to wrap all 3rd party interfaces into custom ones to decouple
libraries from the client code. If an adapter is used, to update the library through an entire
application, only the adapter needs to be updated, as apposed to all the places in the application
where the library gets called.

\subsection{Facade}

\begin{displayquote}
Facade is a structural design pattern that lets you provide a simplified interface to a complex system
of classes, library or framework.  \cite{guru}
\end{displayquote}

Nobody wants to deal with a complex interface, but they are necessary to expose low level options that
are sometimes needed. Wrapping a complex interface in a Facade allows for easy of use the majority of
the time, with the ability to get at the inner workings when necessary. Most plotting libraries are
facades around low level graphics api's or even more plotting libraries. Matplotlib is a python plotting
library that does this. \cite{matplotlib}

\subsection{State}

\begin{displayquote}
Allow an object to alter its behavior when its internal state changes. \cite{guru}
\end{displayquote}

\section{S.O.L.I.D. Principles}

\subsection{\underline{S}ingle Responsibility}

A class/function should do only one thing. In practice this means keep your functions/classes
small, really, really, small. This is using language language constructs to logically organize
code into neat, concise boxes.

\subsection{\underline{O}pen, Closed}

Open for extension, closed for modification. Sub classes can be easily added, but the interface
remains the same. This becomes extremely important with library code, where change the interface
could break all the client code that is calling it.

\begin{displayquote}
``A class should have only one reason to change'' -– Robert Martin
\end{displayquote}

\subsection{\underline{L}iskov Substitution}

The Liskov Substitution principle means that all subclasses can be treated the same as the base class.
This is an 'is a' relationship. The classic example is a rectangle class with setWidth and setHeight methods.
In this case a square is not a rectangle, because you cannot independently set the height and width of a square
like you can a rectangle.

\subsection{\underline{I}nterface Segregation}

Client only depends on methods that it directly uses. This means splitting up large interface into smaller,
more focused ones. This makes them easier to learn/use. No one wants to slog through a interface with 100
methods, when the job could be accomplished with a single call.

\subsection{\underline{D}ependency Inversion}

High level classes should not depend on lower level ones. In practice this means passing in
instances of objects instead of creating them within the constructor. This decouples the ctor
of the lower level object and the height level one. There are frameworks for this, but it can
be done manually.


\bibliographystyle{alpha}
\bibliography{final}

\end{document}
